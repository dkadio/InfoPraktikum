% !TEX encoding = UTF-8 Unicode
%Präambel

%Report für große Doukumente. Dieser ist in Kapitel (\chapter{}) aufgeteilt
%\documentclass[12pt, a4paper, ngerman]{report} 

%Article für normale Doumente
\documentclass[12pt, a4paper, ngerman]{article}

%Deutsche Beschreibungen von generiertem Text (table of contents => Inhaltsverzeichnis)
\usepackage[ngerman]{babel}

%Umlaute
\usepackage[utf8]{inputenc}

%Schriftart Helvetica 
\usepackage[scaled]{helvet}

%Seitenränder
\usepackage{geometry}
%top = Abstand nach oben
%left = Abstand nach links
%right = Abstand nach rechts
%bottom= Abstand nach unten
%heapsep= Abstand zwische Kopfzeile und Text
%footskip= Abstand zwischen Text und Fußzeile
\geometry{a4paper, top=25mm, left=30mm, right=25mm, bottom=30mm, headsep=10mm, footskip=12mm}

%Farben nutzen
\usepackage{xcolor}

%Grafiken einbinden
\usepackage{graphicx}

%Zusätzliche Positionsbefehle
\usepackage{float} 

%Die Einrücktiefe bei einem neuen Absatz
\setlength{\parindent}{0pt}

%Fülltext
\usepackage{blindtext}


%Fuer Zitate	
\PassOptionsToPackage{backend=bibtex}{biblatex}
\usepackage[natbib=true,style=numeric]{biblatex}
\usepackage[babel,german=guillemets]{csquotes}
\bibliography{quellen.bib} 

% Aufnahme von \paragraph in das Inhaltsverzeichnis 
\setcounter{tocdepth}{3}  

%Nummerierung vertiefen, \paragraph kommt mit ins Inhaltsverzeichnis
\setcounter{secnumdepth}{4} 

%Feste Tabellen
\usepackage{tabulary}

%caption für nummerierte Tabellenüberschriften
%booktabs für die Steuerung von Linien
\usepackage{caption, booktabs}

%Package fuer Use Cases
\usepackage{useCases}

%Package, um Attribute zu beschreiben
\usepackage{attribute}

%Package, um PDF Dokumente einzubinden
\usepackage{pdfpages} 


%Eigene Kommandos

\newcommand{\todo}[1]{\fcolorbox{red}{yellow}{ \parbox{0.75\linewidth}{#1}}} %To mark tasks can be done, but are not done 
\newcommand{\later}[1]{\fcolorbox{red}{orange}{\parbox{0.75\linewidth}{#1}}} %To mark tasks in the future. Can't be done now (e.g. missing information)
\newcommand{\checked}[1]{\fcolorbox{green}{green}{\parbox{0.75\linewidth}{#1}}} %To mark tasks, containing checked information
\newcommand{\crap}[1]{\fcolorbox{red}{red}{\parbox{0.75\linewidth}{#1}}} %To mark tasks, being absolutely bullshit

\begin{document}

\tableofcontents 
\newpage

\section{Einleitung}

\section{Use Cases}
Im Folgenden wird beschrieben, welche Use Cases das Programm in seiner Release Version bedienen kann.

\begin{figure}[htbp] 
  \centering
     \includegraphics[width=0.9\textwidth]{Grafiken/primaryUseCases.jpg}
  \caption{Übersicht über die Use Cases}
  \label{fig:uebersichtUseCases}
\end{figure}

\subsection{Automatischer Ablauf}
\subsubsection{Datei einlesen \label{uc:DateiEinlesen}}
\begin{usecase}
	\utitle{Datei einlesen}
	\udescription{Nachdem das Programm gestartet wird, wird die csv Datei mit den Streckeninformationen eingelesen.}
	\uactor{Routenplaner}
	\utrigger{Der Start des Programms}
	\uprecondition{Eine gültige Datei mit den Streckeninformationen muss vorliegen, Der richtige Dateiname und -Pfad muss im Programm angegeben sein}
	\upostcondition{Der Inhalt der Datei befindet sich im Arbeitsspeicher}
\end{usecase}

\subsubsection{Datei verarbeiten \label{uc:DateiVerarbeiten}}
\begin{usecase}
	\utitle{Datei verarbeiten}
	\udescription{Aus der bereits eingelesenen Datei werden Objekte erstellt.}
	\uactor{Routenplaner}
	\utrigger{Die Verarbeitung Use Case \ref{uc:DateiEinlesen} ist abgeschlossen}
	\uprecondition{Use Case \ref{uc:DateiEinlesen}}
	\upostcondition{Die primäre Funktionalität des Routenplaners steht dem User zur Verfügung}
\end{usecase}

\subsubsection{Distanzen berechnen \label{uc:RouteBerechnen}}
\begin{usecase}
	\utitle{Distanzen berechnen}
	\udescription{Der Routenplaner berechnet die kürzesten Wege vom Startpunkt zu allen anderen Punkten}
	\uactor{Routenplaner}
	\utrigger{Use Case \ref{uc:StartpunktWaehlen}} 
	\uprecondition{Use Case \ref{uc:DateiVerarbeiten}}
	\upostcondition{Die Distanzen vom Startpunkt zu allen anderen anderen Punkten sind bekannt}
\end{usecase}
\subsection{Abläufe, die vom User angestoßen werden}
\subsubsection{Startpunkt wählen \label{uc:StartpunktWaehlen}}
\begin{usecase}
	\utitle{Startpunkt wählen}
	\udescription{Der User wählt den Startpunkt der Routenberechnung aus}
	\uactor{User}
	\uprecondition{Use case \ref{uc:DateiVerarbeiten}, Der Startpunkt ist ein gültiger Punkt}
	\upostcondition{Die Route wird berechnet}
\end{usecase}

\subsubsection{Startpunkt ändern \label{uc:StartpunktAendern}}
\begin{usecase}
	\utitle{Startpunkt ändern}
	\udescription{Der User ändert den Startpunkt einer bereits berechneten Route}
	\uactor{User}
	\uprecondition{Use case \ref{uc:DateiVerarbeiten}, Der Startpunkt ist ein gültiger Punkt}
	\upostcondition{Die Route wird neu berechnet}
\end{usecase}

\subsubsection{Zielpunkt ändern \label{uc:ZielpunktAendern}}
\begin{usecase}
	\utitle{Zielpunkt ändern}
	\udescription{Der User trägt einen Zielpunkt ein, bzw. ändert einen bestehenden. }
	\uactor{User}
	\uprecondition{\ref{uc:RouteBerechnen}}
	\upostcondition{Der Routenplaner sucht die kürzeste Route und gibt diese aus (Use Case \ref{uc:RouteAusgeben})}
\end{usecase}

\subsubsection{Route ausgeben \label{uc:RouteAusgeben}}
\begin{usecase}
	\utitle{Route ausgeben}
	\udescription{Die kürzeste Route vom Start- zum Zielpunkt wird ausgegeben}
	\uactor{Routenplaner}
	\uprecondition{Use Case \ref{uc:RouteBerechnen},Use Case \ref{uc:ZielpunktAendern}}
\end{usecase}

\section{Verarbeitete Daten}
Der Routenplaner verarbeitet sämtliche Daten, die im Datensatz vorhanden sind. Da ein Großteil dieser Daten für die Verarbeitung optional ist, werden die notwendigen Daten mit \textit{mandatory} und die optionalen Daten mit \textit{optional} gekennzeichnet. Der Hintergrund, dass die optionalen Daten verarbeitet werden ist, dass optionalen Daten für eine informative Ausgabe benötigt werden können und dem User mehr Komfort bei der Ausgabe bieten können. Die optionalen Daten sind permanent vorhanden und können jederzeit im User Interface abgegriffen und ausgegeben werden.
Die Abbildung \ref{fig:klassendiagrammLokations} zeigt die Hierarchie der Lokationen und deren Attribute.

\begin{figure}[htbp] 
  \centering
     \includegraphics[width=0.3\textwidth]{Grafiken/klassenDiagrammLokations.jpg}
  \caption{Hierarchie der Lokationsklassen}
  \label{fig:klassendiagrammLokations}
\end{figure}

Der originale Datensatz wurde in drei Klassen nachgebildet. Diese entsprechen auch den verschiedenen Location, die in der original Dokumentation des Bundes beschrieben sind.
\subsection{Area Location \label{AreaLocation}}
Eine Area Location, oder Gebietslokation, ist ein grobes Gebiet. Gebiete können beispielsweise Kontinente, Länder, Bundesländer oder Ballungsgebiete sein. Sie stellt den Grundtyp der anderen Lokalionen dar.

\begin{attribut}{Id}
	\drequired{ja}
	\dtype{unsigned int}
	\dmapping{LOCATIONCODE}{Id}
	\dbeschreibung{Eine Id ist einer Lokation eindeutig zugeordnet. Sie wird mit aus dem Datensatz ausgelesen und fest mit dem Objekt verbunden.}
\end{attribut}

\begin{attribut}{typBuchstabe}
	\drequired{nein}
	\dtype{char}
	\dmapping{TYPE}{typBuchstabe + typZahl}
	\dbeschreibung{Dieses Attribut gibt an, um welchen Typ von Lokation es sich handelt. Der ursprüngliche Datentyp besteht aus der Kombination \textless Art der Lokation\textgreater \textless Beschreibung\textgreater. Die Art der Lokation ist ein Buchstabe, der in diesem Attribut gespeichert wird.}
	\dbelegung{A - Gebietslokation, L - Linearlokation, P - Punktlokation }
\end{attribut}

\begin{attribut}{typZahl}
	\drequired{nein}
	\dtype{int}
	\dmapping{TYPE}{typBuchstabe + typZahl}
	\dbeschreibung{Dieses Attribut beschreibt die Lokation genauer. Der ursprüngliche Datentyp besteht aus der Kombination \textless Art der Lokation\textgreater \textless Beschreibung\textgreater. Die genauere Beschreibung der Lokation ist eine Zahl.}
	\dbelegung{Die Zuordnung der Zahlen kann dem Dokument \ref{bundFeinDoku} entnommen werden. Zu beachten ist hierbei\, dass für verschiedene Lokationen andere Zahlenschlüssel gelten.}
\end{attribut}
Die Beschreibung von Lokationen kann weiter herunter gebrochen werden. Die Beschreibung, welche Zahlenwerte hierfür zur Verfügung stehen findet im angehängten Dokument unter \ref{bundFeinDoku}.

\begin{attribut}{feinTyp}
	\drequired{nein}
	\dtype{int}
	\dmapping{SUBTYPE}{feinTyp}
	\dbeschreibung{Der feinTyp beschreibt den Typ einer Lokation genauer. Der Typ A5 beispielsweise steht für ein Gewässer. Wird das Gewässer durch den Feintyp 2 genauer spezifiziert, wird das allgemeine Gewässer zu einem Binnensee.}
	\dbelegung{Siehe \ref{bundFeinDoku}. Dort in der Spalte \glqq Untertyp\grqq~beschrieben}	
\end{attribut}

\begin{attribut}{firstName}
	\drequired{ja}
	\dtype{string}
	\dmapping{FIRST\_NAME}{firstName}
	\dbeschreibung{Der firstName ist der Name der Lokation. Dies kann beispielsweise \glqq Deutschland\grqq~oder \glqq Wallerfangen\grqq~sein. }
\end{attribut}

\begin{attribut}{aktualitaet}
	\drequired{nö}
	\dtype{Aktualitaet*}
	\dmapping{ACTUALITY}{aktualitaet}
	\dbeschreibung{Die aktualitaet beschreibt vermutlich das Erstelldatum der Lokation. In \ref{bundAttributListe} ist dieses Datenfeld nicht dokumentiert. Anhand den aufgetretenen Werten wurde ermittelt, dass es sich dabei um ein Datum handeln muss. Der Wertebereich liegt grob zwischen 2008 und 2013. Die Vermutung, dass dieses Datenfeld das Erstelldatum oder Einpfegedatum der Lokation ist, stützt sich auf den Namen und den Wertebereich. Die Aktualitaet Klasse im Programm stützt sich auf ein \glqq tm *zeit\grqq~struct, ist aber komfortabler zu nutzen.}
\end{attribut}

\begin{attribut}{linLocations}
	\drequired{nope}
	\dtype{vector\textless Linearlokation*\textgreater}
	\dbeschreibung{Jede Gebietslokation hat die Linearlokationen \ref{linearLokation}, die diese Gebietslokation als übergeordnet angegeben hat in einem Vector gespeichert.}
\end{attribut}

\subsection{Linearlokation \label{linearLokation}}

\begin{attribut}{roadNumber}
	\drequired{nein}
	\dtype{string}
	\dmapping{ROADNUMBER}{roadNumber}
	\dbeschreibung{Die roadNumber gibt die Nummer der Straße nach der Einteilung von Bund und Ländern an. Sie ist mit einem führenden Buchstaben und einer Nummer angegeben. Ein Beispiel für eine roadNumber ist \glqq A620\grqq. Neben den bekannten Bezeichnungen ist auch die Bezeichnung \glqq GDD\grqq aufgetreten. Diese Bezeichnung stammt von einem privaten Anbieter. Näheres dazu gibt es im Kapitel \ref{bundGDD}}
\end{attribut}

\begin{attribut}{roadName}
	\drequired{nein}
	\dtype{string}
	\dmapping{ROADNAME}{roadName}
	\dbeschreibung{Der roadName ist der Name einer Straße. Dieses Attribut darf nicht mit \ref{dt:roadNumber} verwechselt werden. Ein Beispiel für einen roadName ist \glqq Metzer Straße\glqq. Diese Straße hat die roadNumber \glqq B406\grqq.}
\end{attribut}

\begin{attribut}{secondName}
	\drequired{nein}
	\dtype{string}
	\dmapping{SECOND\_NAME}{secondName}
	\dbeschreibung{Der secondName ist ein zusätzlicher Name einer Lokation. Ein Beispiel hierfür ist die Straße \glqq GDD12\grqq, \glqq Budapester Straße\grqq, \glqq Georgplatz\grqq, wobei die dritte Angabe der socondName ist.}
\end{attribut}

\begin{attribut}{areaReference}
	\drequired{nein}
	\dtype{Gebietslokation*}
	\dmapping{AREA\_REFERENCE}{areaReference}
	\dbeschreibung{Jede Linear Location hat eine übergeordnete Area Location \ref{AreaLocation}. Dieses Attribut ist ein Pointer auf diese übergeordnete Area Lokation.}
\end{attribut}

\begin{attribut}{negativeOffset}
	\drequired{ja}
	\dtype{Linearlokation*}
	\dmapping{NEGATIVE\_OFFSET}{negativeOffset}
	\dbeschreibung{Der negativeOffset ist grundsätzlich eine vorangehende oder nachfolgende Linear Location. Laut der Beschreibung des Bundes eine Vorgängerlokation bezogen auf die Erfassungsrichtung. Das Attribut ist als Pointer angelegt und erlaubt eine Bewegung durch die Lokationen. Für die Distanzberechnung wurde angenommen, dass es zwischen den Offset eine direkte Verbindung gibt.}
\end{attribut}

\begin{attribut}{positiveOffset}
	\drequired{ja}
	\dtype{Linearlokation*}
	\dmapping{POSITIVE\_OFFSET}{positiveOffset}
	\dbeschreibung{Siehe Attribut \ref{dt:negativeOffset}. Im Gegensatz zum negativeOffset wird der positiveOffset als eine Nachfolgelokation bezogen auf die Erfassungsrichtung beschrieben.}
\end{attribut}

\begin{attribut}{urban}
	\drequired{nein}
	\dtype{bool}
	\dmapping{URBAN}{urban}
	\dbeschreibung{Flag, ob Verkehr städtischen Charakters vorliegt.}
	\dbelegung{0 = nein = false, 1 = ja = true }
\end{attribut} 	
	

\begin{attribut}{intersectioncode}
	\drequired{ja}
	\dtype{Linearlokation*}
	\dmapping{INTERSECTIONSCODE}{intersectioncode}
	\dbeschreibung{Der intersectioncode ist ein Pointer auf eine andere Linear Location, die die aktuelle Linear Location kreuzt. Bei der Distanzberechnung wird sie als potenzieller Nachfolger angenommen. Aus den Datensätzen ist nicht ersichtlich, ob der Intersectioncode tatsächlich eine Verbindung zu der aktuellen Linear Location hat. Näheres dazu steht im Kapitel \ref{Probleme}}
\end{attribut}

\begin{attribut}{interruptsRoad}
	\drequired{nein}
	\dtype{Linearlokation*}
	\dmapping{INTERRUPTS\_ROAD}{interruptsRoad}
	\dbeschreibung{\todo{Hast du noch im Kopf, was das Ding konnte?}}
\end{attribut}

\todo{Macht das hier Sinn, dass man dabei schreibt, dass die Attribute bei der Distanzberechnung ignoriert werden? Eigentlich sind sie ja eh optional 
\begin{attribut}{inPositive}
	\drequired{nein}
	\dtype{bool}
	\dmapping{IN\_POSITIVE}{inPositive}
	\dbeschreibung{inPositive beschreibt, ob die Lotion in Erfassungsrichtung zugänglich ist. Dieses Attribut wird bei der Distanzberechnung ignoriert.}
	\dbelegung{0 = nein = false, 1 = ja = true}
\end{attribut}

\begin{attribut}{outPositive}
	\drequired{nein}
	\dtype{bool}
	\dmapping{OUT\_POSITIVE}{outPositive}
	\dbeschreibung{outPositive beschreibt, ob die Lotion in Erfassungsrichtung verlassen werden kann. Dieses Attribut wird bei der Distanzberechnung ignoriert.}
	\dbelegung{0 = nein = false, 1 = ja = true}
\end{attribut}


\begin{attribut}{inNegative}
	\drequired{nein}
	\dtype{bool}
	\dmapping{IN\_NEGATIVE}{inNegative}
	\dbeschreibung{inNegative beschreibt, ob die Lotion entgegen der Erfassungsrichtung zugänglich ist. Dieses Attribut wird bei der Distanzberechnung ignoriert.}
	\dbelegung{0 = nein = false, 1 = ja = true}
\end{attribut}

\begin{attribut}{outNegative}
	\drequired{nein}
	\dtype{bool}
	\dmapping{OUT\_NEGATIVE}{outNegative}
	\dbeschreibung{outPositive beschreibt, ob die Lotion entgegen der Erfassungsrichtung verlassen werden kann. Dieses Attribut wird bei der Distanzberechnung ignoriert.}
	\dbelegung{0 = nein = false, 1 = ja = true}
\end{attribut}


\begin{attribut}{presentPositive}
	\drequired{nein}
	\dtype{bool}
	\dmapping{PRESENT\_POSITIVE}{presentPositive}
	\dbeschreibung{outPositive beschreibt, ob die Lotion in Erfassungsrichtung vorhanden ist. Dieses Attribut wird bei der Distanzberechnung ignoriert.}
	\dbelegung{0 = nein = false, 1 = ja = true}
\end{attribut}


\begin{attribut}{presentNegative}
	\drequired{nein}
	\dtype{bool}
	\dmapping{PRESENT\_NEGATIVE}{inNegative}
	\dbeschreibung{presentNegative beschreibt, ob die Lotion entgegen der Erfassungsrichtung vorhanden ist. Dieses Attribut wird bei der Distanzberechnung ignoriert.}
	\dbelegung{0 = nein = false, 1 = ja = true}
\end{attribut}}

\begin{attribut}{veraendert}
	\drequired{nein}
	\dtype{int}
	\dmapping{VER\"ANDERT}{veraendert}
	\dbeschreibung{Flag, ob Datensatz bei Aktualisierungslauf gegenüber der Vorgängerversion verändert wurde (nur vor Release erkennbar). Im vorliegenden Datensatz war veraendert in jeder Location == 0}
	\dbelegung{0 - nein, 1 - ja, 3 - löschen}
\end{attribut}


\begin{attribut}{tern}
	\drequired{nein}
	\dtype{bool}
	\dmapping{TERN}{tern}
	\dbeschreibung{Angabe, ob Lokation zum TERN-Netz gehört.}
	\dbelegung{0 = nein = false, 1 = ja = true}
\end{attribut}
	
\begin{attribut}{poldir}
	\drequired{nein}
	\dtype{string}
	\dmapping{POLDIR}{poldir}
	\dbeschreibung{Hinweis auf die nächste zuständige Polizeibehörde.}	
\end{attribut}
	

\begin{attribut}{adminBundesLand}
	\drequired{nein}
	\dtype{string}
	\dmapping{ADMIN\_County}{adminBundesLand}
	\dbeschreibung{Der adminBundesLand gibt an, welches Bundesland für die Bearbeitung dieser Lokation zuständig ist. }
	\dbelegung{BW - Baden-Württemberg, BY - Bayern, BE - Berlin, BB - Brandenburg, HB - Bremen, HH - Hamburg, HE - Hessen, MV - Mecklenburg-Vorpommern, NI - Niedersachsen, NW - Nordrhein-Westfalen, RP - Rheinland-Pfalz, SL - Saarland, SN - Sachsen, ST - Sachsen-Anhalt, SH - Schleswig-Holstein, TH - Thüringen}
\end{attribut}
	

\section{Probleme bei der Routenberechnung \label{Probleme}}

\section{Anhang}
\subsection{Attributliste \label{bundAttributListe}}
\includepdf[pages={1-2}]{Grafiken/RelatedDocuments/Attributliste_zur_LCL12.pdf}

\subsection{Typen und Untertypen der Lokationen \label{bundFeinDoku}}
\includepdf[pages={1-14}]{Grafiken/RelatedDocuments/Typenliste_V08_2010-03-03.pdf}

\subsection{Broschüre eines privaten Anbieters für Straßenverwaltungssoftware \label{bundGDD}}
\includepdf[pages=1-1]{Grafiken/RelatedDocuments/GDD-Strasse_PI.PDF}
\end{document}